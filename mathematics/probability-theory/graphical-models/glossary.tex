\DeclareMathOperator*{\realnumbers}{\mathrm{I\!R}}

% ---- Glossary Entries ----

\longnewglossaryentry{factor}{name=factor, symbol=\ensuremath{\phi}}
{%
  Given a \gls{set of random variables} $\bm{D}$, a factor $\phi$ is a function from $\gls{val}(\bm{D})$ to $\realnumbers$ (not normalized). A factor is called nonnegative if all its entries are nonnegative. $\bm{D}$ is called the \glslink{factor scope}{scope of the factor} and denoted \glssymbol{factor scope}.
}

\longnewglossaryentry{factor scope}{name=factor scope, symbol=\ensuremath{Scope[\phi]}}
{%
  A \gls{set of random variables} $\bm{D}$ of which $\gls{val}(\bm{D})$ is used as the input for the \gls{factor} function. $\bm{D}$ is denoted $Scope[\phi]$.
}

\longnewglossaryentry{i-map}{name=I-map, symbol=\ensuremath{\mathcal{K}}}
{%
  Short for independency map. A graph object $\mathcal{K}$ associated with a \gls{set of independencies} $\mathcal{I}(\mathcal{K})$ is an I-map for a set of indepencencies $\mathcal{I}$ if $\mathcal{I}(\mathcal{K})\subseteq\mathcal{I}$.

  Additionally, $\mathcal{G}$ is and I-map for a \gls{probability distribution} $P$ if $\mathcal{G}$ is an I-map for $\mathcal{I}(P)$.

}

\longnewglossaryentry{markov network}{name=Markov network}
{%
  An undirected graph where the the vertices represent the \glspl{random variable} and the edges correspond to a notion of direct probabilistic interaction between the neighboring variables.\\

  For each edge we create a \gls{factor} containing the two adjacent vertices, and assign values to each of the possible assignment pairs. A global model (probability distribution) is defined by multiplying all factors, normalized by a \gls{partitioning function} as a normailizing constant.

  Also called Markov random field.

}

\longnewglossaryentry{partitioning function}{name=partitioning function, symbol=\ensuremath{Z}}
{%
  A normalizing constant, denoted $Z$, taking the sum of the \glspl{factor} of a \gls{set of random variables} over all its assignments.
}

\longnewglossaryentry{factor product}{name=factor product, symbol=\ensuremath{\psi}}
{%
  The product of two \glspl{factor} $\phi_1(\bm{X},\bm{Y})$ and $\phi_2(\bm{Y},\bm{Z})$, agreeing on $\bm{Y}$ in the following way:
  \begin{equation*}
    \psi(\bm{X},\bm{Y},\bm{Z})=\phi_1(\bm{X},\bm{Y})\cdot\phi_2(\bm{Y},\bm{Z})
  \end{equation*}
  where $\bm{X}$, $\bm{Y}$, and $\bm{Z}$ are \glspl{set of random variables}.
}

\longnewglossaryentry{gibbs distribution}{name=Gibbs distribution, symbol=\ensuremath{P_{\Phi}}}
{%
  A \gls{probability distribution} $P_{\Phi}$ is a Gibbs distribution parameterized by a set of \glspl{factor} $\Phi=\{\phi_1(\bm{D}_1),\dots,\phi_K(\bm{D}_K)\}$ if it is defined as follows:
  \begin{equation*}
    P_{\Phi}(X_1,\dots,X_n)=\frac{1}{Z}\tilde{P}_{\Phi}(X_1,\dots,X_n),
  \end{equation*}
  where
  \begin{equation*}
    \tilde{P}_{\Phi}(X_1,\dots,X_n)=\prod_{i=1}^{m}\phi_i(\bm{D}_i)
  \end{equation*}
  is an unnormalized measure and
  \begin{equation*}
    Z=\sum_{X_1,\ldots,X_n}\tilde{P}_{\Phi}(X_1,\dots,X_n)
  \end{equation*}
  is the \gls{partitioning function}.
}

\longnewglossaryentry{markov network factorization}{name=Markov network factorization}
{%
  A \gls{gibbs distribution} $P_{\Phi}$ with $\Phi=\{\phi_1(\bm{D}_1),\dots,\phi_K(\bm{D}_K)\}$ factorizes over a \gls{markov
network} $\mathcal{H}$ id each $\bm{D}_{k}(k=1,\dots,K)$ is a \gls{complete subgraph} of $\mathcal{H}$.
}

\longnewglossaryentry{conditional random field}{name=conditional random field}
{%
An undirected \gls{graph} whose \glspl{node} correspond to $\bm{Y}\cup\bm{X}$ where $\bm{Y}$ is a set of \glspl{target variable} and $\bm{X}$ is a set of \glspl{observed variable}. The network is annotated with a set of \glspl{factor} $\phi_1(\bm{D}_1),\dots,\phi_m(\bm{D}_m)$ such that each $\bm{D}_i\nsubseteq\bm{X}$. The network encodes a \gls{conditional probability distribution} with
  \begin{equation*}
    P(\bm{Y}\mid\bm{X})=\frac{1}{Z(\bm{X})}\tilde{P}(\bm{Y},\bm{X})
  \end{equation*}
  where
  \begin{equation*}
    \tilde{P}(\bm{Y},\bm{X})=\prod_{i=1}^{m}\phi_i(\bm{D}_i)
  \end{equation*}
  is an unnormalized measure and
  \begin{equation*}
    Z(\bm{X})=\sum_{\bm{Y}}\tilde{P}(\bm{Y},\bm{X})
  \end{equation*}
  is the \gls{partitioning function}.

  Two variables in $\mathcal{H}$ are connected by an (undirected) \gls{edge} whenever they appear together in the \glslink{factor scope}{scope} of some \gls{factor}.
}

\longnewglossaryentry{naive markov model}{name=naive Markov model}
{%
  A \gls{conditional random field} over the binary-valued \glspl{observed variable} $\bm{X}=\{X_1,\dots,X_k\}$ and binary-valued \glspl{target variable} $\bm{Y}=\{Y\}$ with a \glspl{edge potential} between $Y$ and each $X_i$ which are defined via the following \gls{log-linear model}:
  \begin{equation*}
    \phi_i(X_i,Y)=\exp\{ w_i \glssymbol{indicator function} \{ X_i=1,Y=1\}\}
  \end{equation*}
}

\longnewglossaryentry{edge potential}{name=edge potential}
{%
  Given a \gls{graph} $\mathcal{H}$, a set of \glspl{factor} defined as:
  \begin{equation*}
    \{\phi(X_i,X_j):(X_i,X_j)\in\mathcal{H}\}
  \end{equation*}
  Edge potentials are used to define \glspl{pairwise markov network}.

}

\longnewglossaryentry{node potential}{name=node potential}
{%
  Given a \gls{graph} $\mathcal{H}$, a set of \glspl{factor} defined as:
  \begin{equation*}
    \{\phi(X_i):i=1,\dots,n\}
  \end{equation*}
  Node potentials are used to define \glspl{pairwise markov network}.
}

\longnewglossaryentry{pairwise markov network}{name=pairwise Markov network}
{%
  A subclass of \glspl{markov network}. Represents \glspl{probability distribution} where all of the \glspl{factor} are over single variables or pairs of variables. A pairwise Markov network over a graph $\mathcal{H}$ with a set of \glspl{node potential} and \glspl{edge potential}.
}

\longnewglossaryentry{log-linear model}{name=log-linear model}
{%
  A \gls{probability distribution} $P$ is a log-linear model over a \gls{markov network} $\mathcal{H}$ id it is associated with:
  \begin{itemize}
    \item a set of \glspl{feature} $\mathcal{F}=\{f_1(\bm{D}_1),\dots,f_k(\bm{D}_k)\}$, where each $\bm{D}_i$ is a \gls{complete subgraph} in $\mathcal{H}$,
    \item a set od weights $w_1,\dots,w_k$,
  \end{itemize}
  such that $P$ has the \gls{logarithmic representation}:
  \begin{equation*}
    P(X_1,\dots,X_n)=\frac{1}{Z}\exp \Bigg[ -\sum_{i=1}^k w_i f_i(\bm{D}_i) \Bigg] .
  \end{equation*}
}

\longnewglossaryentry{factor graph}{name=factor graph}
{%
  A factor graph $\mathcal{F}$ is an undirected \gls{graph} containing two types of \glspl{node}: Variable nodes and factor nodes. The graph only contains \glspl{edge} between variable nodes and factor nodes. It is parameterized by a set of \glspl{factor}, where each factor node $V_\phi$ is accociated with precisely one factor $\phi$, whose \glslink{factor scope}{scope} is the set of variables that are neghbors of $V_\phi$ in the graph.
}

\longnewglossaryentry{factorization}{name=factorization}
{%
  A \gls{probability distribution} $P$ factorizes over a \gls{factor graph} $\mathcal{F}$ if it can be represented as a set of \glspl{factor} according to the structure of $\mathcal{F}$.
}

\longnewglossaryentry{energy function}{name=energy function,symbol=\ensuremath{\epsilon}}
{%
  A \gls{function} $\epsilon(\bm{D})$ where $\bm{D}$ is a \gls{set of random variables}:
  \begin{equation*}
    \epsilon(\bm{D})=-\ln\phi(\bm{D})
  \end{equation*}
  It is used to rewrite a \gls{factor} $\phi(\bm{D})$:
  \begin{equation*}
    \phi(\bm{D})=\exp(-\epsilon(\bm{D})
  \end{equation*}
  This allows a \gls{logarithmic representation} of a \gls{markov network} that is parameterized using positive \glspl{factor}.
}

\longnewglossaryentry{logarithmic representation}{name=logarithmic representation}
{%
  Using the \gls{energy function} of a set of positive \glspl{factor} of a \gls{markov network} we can represent the \gls{probability distribution} with:
  \begin{equation*}
    P(X_1,\dots,X_n)\propto\exp \Bigg[ -\sum_{i=1}^m \epsilon_i(\bm{D}_i) \Bigg] .
  \end{equation*}
  This ensures that the probability distribution is positive. The logarithmic parameters can take any value along the real line.

}

\longnewglossaryentry{feature}{name=feature}
{%
  Given a subset of \glspl{random variable} $\bm{D}$, a feature $f(\bm{D})$ is a \gls{function} from $\gls{val}(\bm{D})$ to $\realnumbers$. The difference to a \gls{factor} is the missing nonnegativity requirement.
}

\longnewglossaryentry{logistic conditional probability distribution}{name=logistic conditional probability distribution}
{%
  A \gls{naive markov model} with its \glspl{edge potential} defined via the the following \gls{log-linear model}:
  \begin{equation*}
  \phi_i(X_i,Y)=\exp\{w_i\glssymbol{indicator function}\{W_i=1,Y=1\}\}
  \end{equation*}
  and a \gls{node potential} defined as:
  \begin{equation*}
    \phi_0(Y)=\exp\{w_0\glssymbol{indicator function}\{Y=1\}\}.
  \end{equation*}
  Following the equations that define \glspl{conditional random field} we can show that:
  \begin{equation*}
    P(Y=1\mid x_1,\dots,x_k)=\glssymbol{sigmoid function}\Bigg( w_0+\sum_{i=1}^k w_i x_i \Bigg) .
  \end{equation*}
}



% ---- Acronyms ----

\newacronym{crf}{CRF}{\gls{conditional random field}}
\newacronym{mrf}{MRF}{\glslink{markov network}{Markov random field}}
